
\documentstyle{article}

\addtolength{\oddsidemargin}{-0.75in}
\setlength{\textwidth}{6.25in}


\font\bigheadfont=cmss12 at 12truept
\font\medheadfont=cmss10 at 10truept
\font\smallheadfont=cmssq8 at 8truept

\def\uwhead{{\vglue -\voffset \vglue -2.1truecm
\bigheadfont\centerline{UNIVERSITY OF WISCONSIN--MADISON}
\medheadfont\baselineskip=.43truecm\centerline{COMPUTER SCIENCES DEPARTMENT}
\vskip.15truecm
\centerline{CS 564:  DATABASE MANAGEMENT SYSTEMS}
\bigheadfont\baselineskip=1.2em \vskip.25truecm
\centerline{Assignment 6:  Relational Query Optimization}
\centerline{Due: ? day, December ? , 1995, 5 p.m.}
\medheadfont \vskip .2 truecm
\centerline{Instructor:  Raghu Ramakrishnan}
\vskip .2truecm}}

\def\endash{{\the\textfont0--}} %% Need these because PostScript fonts do not
\def\emdash{{\the\textfont0---}}%% make these dashes join together.
\renewcommand{\baselinestretch}{1.2}
\begin{document}

\uwhead

\section{Introduction}

In this assignment, you will carry out a number of exercises involving
the optimization of relational queries using the Minibase optimizer
and visualization tool.  You will not have to write any code, and
your answers should be turned in as a text file.

\section{Available Documentation}

The chapter on optimization and the lecture transparencies
should provide all the background information that you need
on query optimization.
The tool itself has an on-line {\em help} feature, and additional
information is included in this handout as part of exercise 
descriptions.


\section{Getting Started}

See the 564 web page, under ``What's New,'' for specific instructions
on how to run the optimizer visualization tool.

These exercises are based on two catalogs, A and B, which are located in files
{\em catalogA} and {\em catalogB}.  Each exercise says which catalog to use.  
The first
step of each exercise should be to open the appropriate catalog, which you do
by choosing {\em Open} from the {\em Optimizer Catalogs} menu of the 
vizualization tool.

\centerline{\bf WARNING: The visualization tool does not (currently,
and unfortunately) show you which catalog you have open.  Make 
sure you open the correct catalog for each question.}


The two catalogs are based on the same schema.  There are three
relations, {\em Emp, Dept}, and {\em Job}, which represent employees,
departments, and job titles of some organization.  Each of these
relations has an integer field which is used as the primary key for
the relation---{\em empid} for the {\em Emp} relation, {\em dno} for
the {\em Dept} relation, and {\jno} for the {\em Job} relation.  The
{\em Emp} relation has foreign keys referring to {\em Dept} and {\em
Job}, representing each employee's department and job title, and the
{\em Dept} relation has a foreign key referring to {\em Emp},
representing the department's manager.

\section{The Exercises}

\begin{enumerate}
%1. << type 1 question >>
\item
This question gets you to explore ``Level 0'' of the optimizer visualization
display.  This level shows all the access methods for each relation involved
in the query.  If you double-click the LEFT mouse button on one of these
access method boxes, you can also see a list of the relation's attributes
with the range of values each attribute takes on.

Open Catalog A and run the tool on the following query.  
\begin{quote}
        SELECT * FROM emp\\
        WHERE ename = "Doe"
\end{quote}

Answer the following questions, looking at level 0 access methods.
\begin{enumerate}
\item
What are the attributes of the {\em Emp} relation?
\item
How many tuples are in the {\em Emp} relation?
\item
How many different values of the {\em ename} attribute are there in
the {\em Emp} relation?
\item
What is the largest job ID number held by any employee?
\end{enumerate}


\item
%11. << type 1 question >>

Open Catalog B, then run the tool on the following query and answer the
following questions.
\begin{auote}
        SELECT * FROM job\\
        WHERE jno < 100
\end{quote}

\begin{enumerate}
\item
How many leaf pages are there in the index on the {\em jdesc} attribute?
\item
How many leaf pages are there in the index on the {\em jname} attribute?
\item
What can you infer about the relative sizes of the {\em jdesc} and {\em jname}
attributes?
\item
What is the last (alphabetically) job name described in the {\em Job} relation?
\end{enumerate}

\item
%2. << type 1 question >>

Level 1 of the optimizer visualization tool shows plans that consist of
walking a single relation using one of the access methods defined for that
relation.  If you double-click the LEFT mouse button on a level 1 plan (and
higher levels too), its box expands to show you more detailed information
about the plan and the estimates made about it.

Use the same catalog, Catalog A, and the same query as before:
\begin{quote}
        SELECT * FROM emp\\
        WHERE ename = "Doe"
\end{quote}

Answer the following questions, looking at level 1 plans.
\begin{enumerate}
\item
What is the estimated total cost of running the (estimated) best plan?
\item
What is the sorting cost associated with
this plan?  What would the sort cost be for {\em any}
plan for this query?
\item
What is the estimated result cardinality for this plan?
\item
The estimated result cardinality is the number of employees that the
optimizer estimates are named "Doe".  Given your knowledge about
the number of different {\em ename} values (see earlier exercise),
is the estimate the optimizer makes a reasonable one? Explain very briefly.
\end{enumerate}

\item
%3. << type 1 question: levels 0 and 1 >>
If you double-click the RIGHT mouse button on a Level 1 node 
(similarly, for higher
level nodes), the tool opens a separate window and displays the entire plan 
denoted by that node.


Use the same catalog, Catalog A, and the same query as before:
\begin{quote}
        SELECT * FROM emp\\
        WHERE ename = "Doe"
\end{quote}

Once again, open Catalog A and run this same query as in questions 1 and 2.
Answer the following questions, looking at both Level 1 plans and 
Level 0 access methods.
\begin{enumerate}
\item
Find the (estimated) best plan.  Which access method does this plan use?
%<< answer: the ename index >>
\item
In what order would the tuples be returned by this plan?  Why?
%<< in ename order, because the index is ordered >>
\item
Disable the access method used by the best plan by double-clicking the
RIGHT mouse button on its Level 0 box.  (The box will then be displayed
in gray, on a color monitor.)  Which access method does the optimizer
consider to be the best now?
%<< answer: the file scan >>
\item
Compare the new best plan with the other plans still shown.  What is it
about the way the best plan accesses tuples that makes it substantially
cheaper than the others?
%<< answer: the file scan reads many tuples per page, while the other indexes read a page per tuple >>
\end{enumerate}

\item
%4. << type 2 question: show understanding of the difference between levels 0 and 1 >>

Open Catalog A, and run the tool on the following query.
\begin{quote}
        SELECT * FROM dept\\
        WHERE dno < 1500
\end{quote}

Answer the following questions.
\begin{enumerate}
\item
How many Department tuples does the optimizer think there are?  Where on
the tool's display is this information shown?
\item
How many departments whose number is less than 1500 does the optimizer
think there are?  Where on the display does this appear?
\item
How does the optimizer arrive at its estimate of the number of
departments whose number is less than 1500?  That is, what calculations
does it perform, and where does the supporting data come from?
\item
Open Catalog B, then run the same query against this catalog.  {\em (WARNING:
opening the new catalog does NOT automatically re-run the current query.
You must re-run the query manually.)}  What is the new estimate of the
number of departments whose number is less than 1500?  Why is it
different from before?
%<< answer: the range and number of departments is different in this catalog >>
\end{enumerate}


\item
%5. << type 1 question >>

Levels 2 and higher of the optimizer tool show the result of joining a
lower-level plan with an access method.  If you double-click the RIGHT mouse
button on a Level 2 plan, the tool opens a window showing the lower-level
plans and access methods that this plan is made up of.

Open Catalog A, and run the following query.
\begin{quote}
        SELECT * FROM emp, dept\\
        WHERE emp.dno = dept.dno
\end{quote}

Answer the following
questions looking at Level 2 plans.  (NOTE: be sure not to confuse LEVEL 2
with the line NUMBERED 2.  The NUMBER of the line of plans you are to look
at is 3, the top line of the display.)
\begin{enumerate}
\item
What kind of join is the best plan?
\item
What is the estimated total cost of running this plan?
\item
How many tuples does the optimizer estimate will result from this query?
\item
How big are the tuples that result from this query?  (i.e., how many
bytes does each tuple occupy?)
\item
How many pages will be filled by the result tuples?
\item
What access methods does the best plan use?
\end{enumerate}

\item
%6. << type 2 question: show difference between levels 1 and 2 >>

Open Catalog A, and run the tool on this query. 
\begin{quote}
        SELECT ename, dname FROM emp, dept\\
        WHERE emp.dno = dept.dno AND emp.jno < 50
\end{quote}

Answer the following
questions, looking at the best plan.
\begin{enumerate}
\item
What kind of join is the best plan?
%<< index nested loops >>
\item
How many tuples does the optimizer estimate it will retrieve from the
Employee relation?  At what level is this information shown?
\item
Why are the Primary Predicate Selectivity and Predicate Selectivity the
same at this node (the same node as in the previous question)?
\item
What calculation does the optimizer do to arrive at the selectivity
estimate for this node?  Where does the underlying information come from?
\item
How much memory does the optimizer estimate it will take to retrieve the
tuples needed from the Employee relation?
\item
How much memory does the optimizer estimate it will it take to run the
whole query?
\end{enumerate}


\eat{RR let's omit this one too ...
\item
%13. << type 2 question: levels 0 and 1 >>

   Open Catalog B, then run the tool on the following query and answer the
   following questions.

        select * from job
        where jno < 100

   a. Find the "best" plan.  Which access method does this plan use?
<< answer: the file scan >>

   b. In what order would the tuples be returned by this plan?  Why?
<< in random order, because the file scan is not ordered >>

   c. Turn off pruning at level 1 by double clicking the RIGHT mouse button on
      the box to the left of the level 1 plans (the box says: 1 / Plan:job /
      pruning:on).  Then compare the cost formulas for the four plans.  Explain
      for each plan why the cost formula is correct.  That is, describe the
      algorithm implied by the cost formula, and explain why the algorithm
      chosen for each plan is applicable.

end eat
}


\item
%14. << type 2 question: show difference between kinds of joins >>

Open Catalog B, then answer the following questions referring to these
queries:
\begin{quote}
     SELECT ename, dname FROM emp, dept, job\\
     WHERE emp.dno = dept.dno AND emp.jno = job.jno\\
             AND dept.dname = job.jname
\end{quote}
\begin{enumerate}
\item
Run this query and describe the best plan: list the joins and access methods
it uses, and the order in which the relations are joined.
%<< 2 index nested loops >>
\item
Modify the query by replacing the last `=' (comparing names) with `$<$', and
optimize it again.  Describe the best plan.
%<< page-oriented nested loops joining job and dept, hash join for emp >>
\item
Explain why there is such a big difference in the plans produced for the query
and for its (slightly) modified version.
\end{enumerate}


\item
%7. << type 3 question: differences between different join methods >>

>From the {\em Switches \/ Toggle Joins} menu of the optimizer visualization tool,
you can disable and re-enable the different kinds of joins that the
optimizer will consider.

Also, if you double-click the RIGHT mouse button on the box that labels any
row of the tool's display, you will turn off (and back on) the pruning of plans
that the optimizer estimates are not the `least expensive'.  (Remember that
for each subset of relations, for each ordering of generated tuples,
only the least cost plan is retained.  The toggling referred to here
controls this pruning; it should be used with care since the number
of retained plans can increase rapidly without pruning!)

Run the following variant of the query from the previous
exercise, using Catalog A as before.  
\begin{quote}
        SELECT ename, dname FROM emp, dept\\
        WHERE emp.dno = dept.dno AND emp.jno < 500
\end{quote}

Now suppose you knew that the run-time environment for this query
was limited to 10 pages of memory.  Find the best (left-deep) plan that fits
this requirement and describe its join method, access methods, and total cost.

%   << answer: index nested loops.  you find this by turning off sort-merge and hash joins. >>

\item

%8. << type 2 question: how index-only plans work >>

>From the {\em Switches} menu it is possible to disable {\em index-only} plans.  
If such plans are disabled, the optimizer will only generate plans that
always retrieve tuples from the underlying relation, without checking to
see if the index search key (and therefore the data entries in the index)
contain all necessary information.

Run the tool on the following query, which lists the number of employees that
hold each job if that number is greater than 5.  Use Catalog A.
\begin{quote}
	SELECT jno, count(*) FROM emp\\
	GROUP BY jno\\
	HAVING count(*) > 5
\end{quote}

\begin{enumerate}
\item
Why is this query a candidate for an index-only plan?
\item
What is the estimated cost of this query when index-only plans are allowed?
\item
You can infer the algorithm that the optimizer assumes by the cost
formula it displays.  Describe the algorithm assumed by the optimizer for
the best plan when index-only plans are allowed.
%RR We should find a way to make this explicit in the displayed plan.
\item
What is the cost of this query when index-only plans are disallowed?
\item
Describe the algorithm assumed by the optimizer for the best plan when
index-only plans are disallowed.
\end{enumerate}

\item
%9. << type 3 question: modify a query to get a result >>

Run the following query using Catalog A.
\begin{quote}
	SELECT * FROM emp, dept\\
	WHERE emp.dno = dept.dno AND emp.jno < 500
\end{quote}

\begin{enumerate}
\item
Find the largest number such that replacing the 500 with it in the above
query makes the best plan into an Index Nested Loops join with a file
scan for the {\em Emp} table.
\item
Find the largest number such that replacing the 500 with it makes the
best plan into an Index Nested Loops join using the index on the {\em jno}
column of the {\em Emp} table.
\item
Repeat the previous two questions using Catalog B.
%<< This time, the two numbers are right next to each other >>
\end{enumerate}


\item
%10. << type 3 question: modify a catalog to get a particular result >>

Consider the following query, as run against Catalog B.  The query lists the
number of employees hired before their manager for each of the first 100 departments.
\begin{quote}
	SELECT M.ename, M.empid, count(*)\\
	FROM emp E, emp M, dept D
	WHERE E.dno = D.dno AND D.mgrid = M.empid
		AND E.dno <= 100 AND E.empid < M.empid
	GROUP BY M.empid, M.ename

\begin{enumerate}
\item
Why does the best plan not use the index on {\em E.dno}?
%<< because it is unordered, so ``dno<=100'' can't readily be determined from
%   that index.  Also, you need other values from the tuple, so the other
%   reason it might be of benefit, index-only, doesn't apply >>
%RR above explanation is incomplete, at best --- LB: how about now?
\item
Make a copy of Catalog B and modify your copy so that the best plan
uses an index on {\em E.dno} when run against your modified catalog.
%<< make the dno index into a btree >>
\item
Find one single access method such that, when you disable it, the best
plan uses only file scans.
%<< the M.empid index is the one >>
\end{enumerate}

\eat{RR: let's think this one over ...
\item

%12. << type 1 question >>

   Open Catalog B, then answer the following questions referring to these
   queries:

    (i)   select jname from job
          where jno < 101

    (ii)  select jname from job
          where jname >= "B" and jname < "C"

    (iii) select jname from job
          where jno < 101 or (jname >= "B" and jname < "C")


   a. Run the tool on query (i).  What is the predicate selectivity of the best
      plan?  What is the estimated cardinality?
<< 50 >>

   b. Run the tool on query (ii).  What is the predicate selectivity of the
      best plan?  What is the estimated cardinality?
<< 9 >>

   c. Run the tool on query (iii).  What is the predicate selectivity of the
      best plan?  What are the individual selectivities of the two predicates?
      What is the estimated cardinality?
<< 42 >>

   d. What range would you have expected the estimated cardinality of query
      (iii) to be in, if all you were given was the estimates for queries (i)
      and (ii)?
<< 50 .. 59 >>

end of eat
}

\end{enumerate}

\section{What to Turn In, and When}

Mail a text version of your answers to the TAs.
The assignment is due at 5PM on December ? .  The solution will be made
public after that time,  and solutions turned  in after that time will
{\bf not} receive any credit.   So make sure to  turn in whatever  you
have at that time.

\end{document}



