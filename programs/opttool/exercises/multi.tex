
\documentstyle{article}

\addtolength{\oddsidemargin}{-0.75in}
\setlength{\textwidth}{6.25in}


\font\bigheadfont=cmss12 at 12truept
\font\medheadfont=cmss10 at 10truept
\font\smallheadfont=cmssq8 at 8truept

\def\uwhead{{\vglue -\voffset \vglue -2.1truecm
\bigheadfont\centerline{UNIVERSITY OF WISCONSIN--MADISON}
\medheadfont\baselineskip=.43truecm\centerline{COMPUTER SCIENCES DEPARTMENT}
\vskip.15truecm
\centerline{CS 564:  DATABASE MANAGEMENT SYSTEMS}
\bigheadfont\baselineskip=1.2em \vskip.25truecm
\centerline{Relational Query Optimization}
\medheadfont \vskip .2 truecm
\centerline{Instructor:  Raghu Ramakrishnan}
\vskip .2truecm}}

\def\endash{{\the\textfont0--}} %% Need these because PostScript fonts do not
\def\emdash{{\the\textfont0---}}%% make these dashes join together.
\renewcommand{\baselinestretch}{1.2}
\begin{document}

\uwhead

\section{Introduction}

The following exercises will serve as a starting point for understanding
query plan selection and optimization on relations whose indexes are
composed of an {\em ordered} sequence of a subset of the fields of that
relation---we will simply such indexes {\em multi-column}.

As before, you will not have to write any code, and your answers should
be turned in as a text file.  Please be brief with your answers (a single
line of prose should be sufficient in almost all cases).


\section{Getting Started}

See the 564 web page, under ``What's New,'' for specific instructions
on how to run the optimizer visualization tool.

All of the following exercises are based on the catalog D which is located
in the file {\em catalogD}.  Once you have opened this catalog, you can
use it for all of the multi-column index exercises.

The first thing you should do is to take a look at the catalog file
{\em catalogD}.  Note that it is much more complicated than the simple
{\em catalogA} and {\em catalogB} files.  This is because {\em catalogD}
can be used in a realistic database---the special relations {\em relcat},
{\em attrcat} and {\em indcat} allow the catalog to describe itself and
the rest of the relations in the database.  However, you will not be
using any of these special relations in this exercise.

The second thing you should note is the way multi-column indexes are
specified---via the use of percent signs.  The order in which the field
names are specified is also important (and this will be demonstrated in
the exercises).

\section{The Relations}

CatalogD has five incarnations of the well-known and widely-used {\em emp}
relation.  Each of the Emp relations has its indexes built on either two
or three of its fields ({\em ename}, {\em title} and {\em dname}).


\section{The Exercises}

\begin{enumerate}
%1. << Simple Query >>
\item
This question gets you to explore what query plans are produced when
your query does not use all of the columns of a multi-column index, and when
your query does use all the fields.

Type in the following query.  
\begin{quote}
        SELECT * FROM emp1\\
        WHERE ename = "Dilbert"
\end{quote}

Answer the following questions looking at Level 1 plans.
\begin{enumerate}
\item
What query plans are produced for this query?
\item
What would be the cost of executing the query?
\item
What fields of the relation are used in the query plan?
\end{enumerate}

Now type in the following query.
\begin{quote}
        SELECT * FROM emp1\\
        WHERE ename = "Dilbert" AND title = "Bozo"
\end{quote}

Answer the following questions looking at Level 1 plans.
\begin{enumerate}
\item
What query plans are produced for this query?
\item
What would be the cost of executing the query?
\item
What fields of the relation are used in the query plan?
\end{enumerate}


\item
%2. << Index-Only Query >>

In the previous question, the queries we used selected the entire tuple
from the relation.  However, if our {\em select} clause projected only
those fields which are already a part of the index, then the query does
not need to access the datapages of the relation.  This question gets you
to explore how the query plans and their costs change when you specify
index-only queries.

This question also demonstrates the importance of column-order in the
search key of a multi-column index.

Type in the following query.
\begin{auote}
        SELECT ename, title FROM emp4\\
        WHERE dname = "Programming" AND ename = "Dogbert"
\end{quote}

Answer the following questions looking at Level 1 plans.
\begin{enumerate}
\item
What query plans are produced for this query?
\item
What would be the cost of executing the query?
\end{enumerate}

Now type in the following query.
\begin{auote}
        SELECT ename, title FROM emp4\\
        WHERE dname = "Programming" AND title = "Hacker"
\end{quote}

Answer the following questions looking at Level 1 plans.
\begin{enumerate}
\item
What query plans are produced for this query?
\item
What would be the cost of executing the query?
\item
Is there any effect of specify different columns in the query plans
and cost?
\end{enumerate}


\item
%3. << Sorting Query >>

This question gets you to explore the difference between queries that
require sorting or not.

Type in the following query.
\begin{quote}
        SELECT title, count(*) FROM emp3\\
        GROUP BY title
\end{quote}

Answer the following questions, looking at level 1 plans.
\begin{enumerate}
\item
Is the query plan index-only?  What indexes are used?
\item
Is there a primary predicate?
\item
Is sorting required?
\end{enumerate}

Now type in the following query.
\begin{quote}
        SELECT title, count(*) FROM emp1\\
        GROUP BY title
\end{quote}

Answer the following questions, looking at level 1 plans.
\begin{enumerate}
\item
Is the query plan index-only?  What indexes are used?
\item
Is there a primary predicate?
\item
Is sorting required?
\end{enumerate}

\end{enumerate}

\section{What to Turn In, and When}

Mail a text version of your answers to the TAs.  The solution will be
made public after that time,  and solutions turned  in after that time
will {\bf not} receive any credit.   So make sure to  turn in whatever
you have at that time.

\end{document}
