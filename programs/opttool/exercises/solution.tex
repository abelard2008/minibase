\documentstyle{article}
\newcommand{\eat}[1]{}

\addtolength{\oddsidemargin}{-0.75in}
\setlength{\textwidth}{6.25in}


\font\bigheadfont=cmss12 at 12truept
\font\medheadfont=cmss10 at 10truept
\font\smallheadfont=cmssq8 at 8truept

\def\uwhead{{\vglue -\voffset \vglue -2.1truecm
\bigheadfont\centerline{UNIVERSITY OF WISCONSIN--MADISON}
\medheadfont\baselineskip=.43truecm\centerline{COMPUTER SCIENCES DEPARTMENT}
\vskip.15truecm
\centerline{CS 564:  DATABASE MANAGEMENT SYSTEMS}
\bigheadfont\baselineskip=1.2em \vskip.25truecm
\centerline{Solutions to Assignment 7:  Relational Query Optimization}
\medheadfont \vskip .2 truecm
\centerline{Instructor:  Raghu Ramakrishnan}
\vskip .2truecm}}

\def\endash{{\the\textfont0--}} %% Need these because PostScript fonts do not
\def\emdash{{\the\textfont0---}}%% make these dashes join together.
\renewcommand{\baselinestretch}{1.2}
\begin{document}

\uwhead

\section{The Exercises}

\begin{enumerate}
%1. << type 1 question >>
\item
Open Catalog A and run the tool on the following query.  
\begin{quote}
        SELECT * FROM job\\
        WHERE jname = "Boss"
\end{quote}

Answer the following questions, looking at level 0 access methods.
\begin{enumerate}
\item
What are the attributes of the {\em Job} relation?

Answer: {\em jno}, {\em jname} and {\em jdesc}.

\item
How many tuples are in the {\em Job} relation?

Answer: 2000.

\item
How many different values of the {\em jname} attribute are there in
the {\em Job} relation?

Answer: 2000.

\item
What is the largest job ID number held by any employee?

Answer: 2500.
\end{enumerate}


\item
%2. << type 1 question >>
Use the same catalog, Catalog A, and the following query:
\begin{quote}
        SELECT * FROM emp\\
        WHERE ename = "Doe"
\end{quote}

Answer the following questions, looking at level 1 plans.
\begin{enumerate}
\item
What is the estimated total cost of running the (estimated) best plan?

Answer: 4.02

\item
What is the sorting cost associated with this plan?  What would the sort
cost be for {\em any} plan for this query?

Answer: 0. All plans have zero sort-cost since there is no grouping
demanded by the query.

\item
What is the estimated result cardinality for this plan?

Answer: 1.

\item
The estimated result cardinality is the number of employees that the
optimizer estimates are named "Doe".  Given your knowledge about the
number of different {\em ename} values (see earlier exercise), is the
estimate the optimizer makes a reasonable one? Explain very briefly.

Answer: Number of tuples is 40,000.  Number of distinct keys is 39,000.
Approx. one key per tuple.  Hence result cardinality of 1 is reasonable.
\end{enumerate}


\item
%3. << type 1 question: levels 0 and 1 >>
Use the same catalog, Catalog A, and the same query as before:
\begin{quote}
        SELECT * FROM emp\\
        WHERE ename = "Doe"
\end{quote}

Answer the following questions, looking at both Level 1 plans and Level
0 access methods.

\begin{enumerate}
\item
Find the (estimated) best plan.  Which access method does this plan use?

Answer: BIndex on ename.

\item
In what order would the tuples be returned by this plan?  Why?

Answer: In ename order, since a BIndex is ordered.

\item
Disable the access method used by the best plan by double-clicking the
RIGHT mouse button on its Level 0 box.  (The box will then be displayed
in gray, on a color monitor.)  Which access method does the optimizer
consider to be the best now?

Answer: FileScan.

\item
Compare the new best plan with the other plans still shown.  What is it about the way the best
plan acces
ses tuples that makes it substantially cheaper than the others?

Answer: FileScan reads each page only once and processes many tuples.
Other indexes (being unclustered) may read a page multiple times and
process only one tuple per page.
\end{enumerate}


\item
%4. << type 2 question: show understanding of the difference between levels 0 and 1 >>
Open Catalog A, and run the tool on the following query.
\begin{quote}
        SELECT * FROM job\\
        WHERE (jno $>$ 100) AND (jno $<$ 1000)
\end{quote}

Answer the following questions.
\begin{enumerate}
\item
How many Job tuples does the optimizer think there are?  Where on the
tool's display is this information shown?

Answer: 2000.  Each Level 0 box shows this.

\item
How many jobs whose number is greater than 100 and less than 1000 does
the optimizer think there are?  Where on the display does this appear?

Answer: 767.  Appears in Level 1 boxes.

\item
How does the optimizer arrive at its estimate of the number of jobs whose
number satisfies the predicate in the query?  That is, what calculations
does it perform, and where does the supporting data come from?

Answer: Reduction factor is computed as the ratio of the range of matching
tuples to the range of all tuples.  Number of matching tuples is product
of reduction factor and the number of all tuples in relation.

\item
Open Catalog B, then run the same query against this catalog.  {\em
(WARNING: opening the new catalog does NOT automatically re-run the
current query.  You must re-run the query manually.)}  What is the new
estimate of the number of jobs whose number is greater than 100 and less
than 1000?  Why is it different from before?

Answer: 100.  The range and number of job tuples are different.
\end{enumerate}


\item
%5. << type 1 question >>
Open Catalog A, and run the following query.

\begin{quote}
        SELECT * FROM emp, dept\\
        WHERE emp.dno = dept.dno
\end{quote}

Answer the following questions looking at Level 2 plans.  (NOTE: be sure
not to confuse LEVEL 2 with the line NUMBERED 2.  The NUMBER of the line
of plans you are to look at is 3, the top line of the display.)

\begin{enumerate}
\item
What kind of join is the best plan?

Answer: Hash Join.

\item
What is the estimated total cost of running this plan?

Answer: 6356.

\item
How many tuples does the optimizer estimate will result from this query?

Answer: 40,000.

\item
How big are the tuples that result from this query?  (i.e., how many
bytes does each tuple occupy?)

Answer: 90.

\item
How many pages will be filled by the result tuples?

Answer: 3516.

\item
What access methods does the best plan use?

Answer: FileScan on dept and FileScan on emp.
\end{enumerate}


\item
%6. << type 2 question: show difference between levels 1 and 2 >>
Open Catalog A, and run the tool on this query. 
\begin{quote}
        SELECT ename, jname FROM emp, job\\
        WHERE emp.jno = job.jno AND emp.sal $>$ 20000
\end{quote}

Answer the following questions, looking at the best plan.
\begin{enumerate}
\item
What kind of join is the best plan?

Answer: Hash join.

\item
How many tuples does the optimizer estimate it will retrieve from the
Employee relation?  At what level is this information shown?

Answer: 40,000. Level-1 boxes.

\item
Why are the Primary Predicate Selectivity and Predicate Selectivity the
same at this node (the same node as in the previous question)?

Answer: Trick question.  There is no primary predicate when indexing the
Employee relation on the {\em sal} attribute.  However, both the primary
reduction factor (RF(Primary)) and predicate reduction factor (RF0)
are identical (equal to 1.0).  This is because a BIndex is available
on {\em sal} that can be used to retrieve matching tuples from the
Employee relation.

\item
What calculation does the optimizer do to arrive at the selectivity
estimate for this node?  Where does the underlying information come from?

Answer: Optimizer estimates reduction factor (assuming uniform 
distribution of attribute values).  Product of reduction factor and
number of tuples in relation give selectivity of an access plan.
Underlying information comes from the catalog.

\item
How much memory does the optimizer estimate it will take to retrieve the
tuples needed from the Employee relation?

Answer: 2 buffer pool pages.

\item
How much memory does the optimizer estimate it will take to run the
whole query?

Answer: 45 buffer pool pages.
\end{enumerate}


\item
%14. << type 2 question: show difference between kinds of joins >>
Open Catalog B, then answer the following questions referring to these
queries:
    
\begin{quote}
     SELECT ename, dname FROM emp, dept, job\\
     WHERE emp.dno = dept.dno AND emp.jno = job.jno\\
             AND dept.dname = job.jname
\end{quote}
    
\begin{enumerate}
\item
Run this query and describe the best plan: list the joins and access
methods it uses, and the order in which the relations are joined. 
    
Answer: FileScan on job (index nested) joined with BIndex on dept.dname.
Result (index nested) joined with Hash on emp.dno.
    
\item
Modify the query by replacing the last `=' (comparing names) with `$<$',
and optimize it again.  Describe the best plan.
    
Answer: FileScan on job (page oriented nested loop) joined with FileScan
on dept.  Result (hash) joined with FileScan on emp. 
    
\item
Explain why there is such a big difference in the plans produced for
the query and for its (slightly) modified version.
    
Answer: Many more tuples satisfy range queries than equality queries.
Nested loop scans can no longer use only the indexes.  Hence page oriented
nested loops are used to answer the name comparison sub-query.
\end{enumerate}


\item
%8. << type 2 question: how index-only plans work >>
Run the tool on the following query, which lists the number of employees
that hold each job if that number is less than 10.  Use Catalog A.

\begin{quote}
	SELECT jno, count(*) FROM emp\\
	GROUP BY jno\\
	HAVING count(*) $<$ 10
\end{quote}

\begin{enumerate}
\item
Why is this query a candidate for an index-only plan?

Answer: BIndex available on emp.jno.  Query can be answered without
retrieving any of the underlying tuples.

\item
What is the estimated cost of this query when index-only plans are
allowed?

Answer: 642.

\item
You can infer the algorithm that the optimizer assumes by the cost
formula it displays.  Describe the algorithm assumed by the optimizer
for the best plan when index-only plans are allowed.

Answer: Use BIndex to scan emp.  Since index is ordered ascending,
compute count on the fly.  Select jno and count values when count is
greater than 5.

\item
What is the cost of this query when index-only plans are disallowed?

Answer: 2628.

\item
Describe the algorithm assumed by the optimizer for the best plan when
index-only plans are disallowed.

Answer: Use FileScan to get all tuples.  Sort to cluster jnos.
Compute count and select.
\end{enumerate}


\item
%9. << type 3 question: modify a query to get a result >>
Run the following query using Catalog A.
\begin{quote}
	SELECT * FROM emp, job\\
	WHERE emp.jno = job.jno AND emp.dno $<$ 2000
\end{quote}

\begin{enumerate}
\item
Find the largest number such that replacing the 2000 with it in the
above query makes the best plan into an Index Nested Loops join with a
file scan for the {\em Emp} table.

Answer: 1195.

\item
Find the largest number such that replacing the 2000 with it makes the
best plan into an Index Nested Loops join using the index on the {\em
dno} column of the {\em Emp} table.

Answer: Trick question.  This will never happen.  Note that the index
on {\em dno} is a Hash index, which is useless for answering range and
non-equality queries.

\item
Repeat the previous two questions using Catalog B.

Answer: 2 (for part a).  Note that any larger value leads to page oriented
loops.  As before, there is no answer for part b of this question.

\end{enumerate}


\item
%7. << type 3 question: differences between different join methods >>

Run the following variant of the query from the previous exercise, using
Catalog A as before.
\begin{quote}
        SELECT ename, dname FROM emp, dept\\
        WHERE emp.dno = dept.dno AND emp.jno < 500
\end{quote}

Now suppose you knew that the run-time environment for this query was
limited to 10 pages of memory.  Find the best (left-deep) plan that
fits this requirement and describe its join method, access methods, and
total cost.

Answer: Index nested loop joins require 3 buffer pool pages.  Sort merge
and Hash join plans require more than 10.  Turn off sort merge and
hash joins.


\item
%10. << type 3 question: modify a catalog to get a particular result >>

Consider the following query, as run against Catalog B.  The query lists
the number of employees hired before their manager for each of the first
100 departments.

\begin{quote}
	SELECT M.ename, M.empid, count(*)\\
	FROM emp E, emp M, dept D
	WHERE E.dno = D.dno AND D.mgrid = M.empid
		AND E.dno $<=$ 100 AND E.empid $<$ M.empid
	GROUP BY M.empid, M.ename
\end{quote}
\begin{enumerate}
\item
Why does the best plan not use the index on {\em E.dno}?

Answer: {\em E.dno} has a Hash index which cannot be efficiently used
to answer range queries.

\item
Make a copy of Catalog B and modify your copy so that the best plan
uses an index on {\em E.dno} when run against your modified catalog.

Answer: Change the index on {\em E.dno} into a BIndex.

\item
Find one single access method such that, when you disable it, the best
plan uses only file scans.

Answer: The index on {\em M.empid} is the critical one.

\end{enumerate}

\end{enumerate}

\section{Multi-Column Index Exercises}

\begin{enumerate}
%1. << Simple Query >>
\item
Type in the following query.  
\begin{quote}
        SELECT * FROM emp1\\
        WHERE ename = "Dilbert"
\end{quote}

Answer the following questions looking at Level 1 plans.
\begin{enumerate}
\item
What query plans are produced for this query?

Answer: BIndex on {\em emp1.ename, emp1.title}.

\item
What would be the cost of executing the query?

Answer: 7.24.

\item
What fields of the relation are used in the query plan?

Answer: All fields ({\em ename}, {\em title} and {\em dname}.

\end{enumerate}

Now type in the following query.
\begin{quote}
        SELECT * FROM emp1\\
        WHERE ename = "Dilbert" AND title = "Bozo"
\end{quote}

Answer the following questions looking at Level 1 plans.
\begin{enumerate}
\item
What query plans are produced for this query?

Answer: BIndex on {\em emp1.ename, emp1.title}.

\item
What would be the cost of executing the query?

Answer: 7.24.

\item
What fields of the relation are used in the query plan?

Answer: All fields ({\em ename}, {\em title} and {\em dname}.

Trick second part question.  Although the reduction factors change, the
query touches so few pages that the cost of executing the more selective
query does not change.
\end{enumerate}


\item
%2. << Index-Only Query >>
Type in the following query.
\begin{quote}
        SELECT ename, title FROM emp4\\
        WHERE dname = "Programming" AND ename = "Dogbert"
\end{quote}

Answer the following questions looking at Level 1 plans.
\begin{enumerate}
\item
What query plans are produced for this query?

Answer: BIndex on {\em dname,title, ename}.

\item
What would be the cost of executing the query?

Answer: 6.

\end{enumerate}

Now type in the following query.
\begin{quote}
        SELECT ename, title FROM emp4\\
        WHERE dname = "Programming" AND title = "Hacker"
\end{quote}

Answer the following questions looking at Level 1 plans.
\begin{enumerate}
\item
What query plans are produced for this query?

Answer: BIndex on {\em dname,title, ename}.

\item
What would be the cost of executing the query?

Answer: 6.

\item
Is there any effect of specify different columns in the query plans
and cost?

Answer: No.  Selection using {\em dname} does not change the reduction
factors.
\end{enumerate}


\item
%3. << Sorting Query >>
Type in the following query.
\begin{quote}
        SELECT title, count(*) FROM emp3\\
        GROUP BY title
\end{quote}

Answer the following questions, looking at level 1 plans.
\begin{enumerate}
\item
Is the query plan index-only?  What indexes are used?

Answer: Yes. BIndex on {\em title, ename}.

\item
Is there a primary predicate?

Answer: No.

\item
Is sorting required?

Answer: No.

\end{enumerate}

Now type in the following query.
\begin{quote}
        SELECT title, count(*) FROM emp1\\
        GROUP BY title
\end{quote}

Answer the following questions, looking at level 1 plans.
\begin{enumerate}
\item
Is the query plan index-only?  What indexes are used?

Answer: Yes. BIndex on {\em ename, title}.

\item
Is there a primary predicate?

Answer: No.

\item
Is sorting required?

Answer: No.

\end{enumerate}

\end{enumerate}

\end{document}
